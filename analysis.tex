\section*{ANALYSIS}\label{analysis}
\hypertarget{fnpf}{%
\subsection*{FNPF}\label{fnpf}}
The FNPF results have the benifit of having all the three states:
$\phi$, $\dot{\phi}$ and $\ddot{\phi}$. This means that these time
series can be inserted into the differential equation
(Eq.\ref{eq:eqroll_decay_equation_quadratic_a}) and the
parameters of the model can be estimated using Ordinary Least Square
method (OLS), solving the following regression:
\begin{equation}
y = X \cdot \beta + \epsilon
\label{ols}
\end{equation}
where:
\begin{itemize}
\item $y$ is the dependent variable (also called *label*).
\item $\beta$ is a vector with the regressed parameters.
\item $X$ is a matrix containing the independent variables (also called *features*).
\end{itemize}
The roll decay equation can be expressed as a linear regression with
\begin{itemize}
\item $y$ : the roll angle acceleration $\ddot{\phi}$
\item $\beta$ : contains all the parameters : $B_1$, $B_2$, $C_1$...
\item $X$ : contains all the time varying features such as: $| \dot{\phi} | \dot{\phi} $ etc.
\end{itemize}
\begin{equation}
\begin{aligned}
- \ddot{\phi} = B_{1A} \dot{\phi} + B_{2A} \left|{\dot{\phi}}\right| \dot{\phi} + B_{3A} \dot{\phi}^{3} + C_{1A} \phi + C_{3A} \phi^{3} \\ + C_{5A} \phi^{5}
\end{aligned}
\label{acceleration_equation_cubic}
\end{equation}
\begin{equation}
y = B_{1A} x_{2} + B_{2A} x_{4} + B_{3A} x_{3} + C_{1A} x_{1} + C_{3A} x_{5} + C_{5A} x_{6}
\label{acceleration_equation_cubic_x}
\end{equation}\displaystyle \left[\begin{matrix}C_{1A} & B_{1A} & B_{3A} & B_{2A} & C_{3A} & C_{5A}\end{matrix}\right]$
$\displaystyle \left[\begin{matrix}x_{1}\\x_{2}\\x_{3}\\x_{4}\\x_{5}\\x_{6}\end{matrix}\right] = \left[\begin{matrix}\phi{\left(t \right)}\\\frac{d}{d t} \phi{\left(t \right)}\\\left(\frac{d}{d t} \phi{\left(t \right)}\right)^{3}\\\left|{\frac{d}{d t} \phi{\left(t \right)}}\right| \frac{d}{d t} \phi{\left(t \right)}\\\phi^{3}{\left(t \right)}\\\phi^{5}{\left(t \right)}\end{matrix}\right]$
\begin{equation}
y = - \ddot{\phi}
\label{eq_y}
\end{equation}
\begin{table}[H]
\scriptsize
\center
\caption{Parameters estimation}
\label{tab:parameters}
\begin{tabular}{|l|l|l|l|l|}
\hline\addlinespace
coeff & mean & pvals & conf_lower & conf_higher\\
C_1A & -6.116 & 0.0 & -6.118 & -6.114\\
\hlineB_1A & -0.016 & 0.0 & -0.018 & -0.014\\
B_3A & -0.098 & 0.0 & -0.124 & -0.072\\
B_2A & 0.062 & 0.0 & 0.047 & 0.076\\
C_3A & 5.522 & 0.0 & 5.236 & 5.807\\
C_5A & -254.093 & 0.0 & -263.264 & -244.923\\
\hline
\end{tabular}
\end{table}
\begin{figure}[H]
\begin{center}\includegraphics[width = 0.95\textwidth]{figures/output_42_0.pdf}\end{center}
\vspace{-0.7cm}
\caption{}
\label{fig:}
\end{figure}
$\displaystyle \left[\begin{matrix}x_{1}\\x_{2}\end{matrix}\right] = \left[\begin{matrix}\phi{\left(t \right)}\\\frac{d}{d t} \phi{\left(t \right)}\end{matrix}\right]$
\begin{figure}[H]
\begin{center}\includegraphics[width = 0.95\textwidth]{figures/output_44_0.pdf}\end{center}
\vspace{-0.7cm}
\caption{}
\label{fig:}
\end{figure}
\begin{table}[H]
\scriptsize
\center
\caption{Parameters estimation}
\label{tab:parameters}
\begin{tabular}{|l|l|l|l|l|}
\hline\addlinespace
coeff & mean & pvals & conf_lower & conf_higher\\
C_1A & -6.101 & 0.0 & -6.101 & -6.1\\
\hlineB_1A & -0.007 & 0.0 & -0.007 & -0.007\\
\hline
\end{tabular}
\end{table}
