\section*{CONCLUSIONS}\label{conclusions}
The parameters in a cubic and a linear roll decay motion model have been
estimated by fitting a linear regression with ordinary least square fit
to roll decay tests time series at 0 knots ship speed. The regression
was validated by comparing simulated roll signals (including the
regressed parameters) with the original roll signals. The coefficient of
determination $R^2$ was slightly higher for the cubic model
(suggesting higher accuracy) compared to the linear model for the
investigation of the FNPF data.
For the model test data, a numerical differentiation was used to
estimate the roll velocity and acceleration, as these signals were
otherwise missing from the model test data. The numerical estimates for
these signals were unfortunatelly very noisy. The residuals from the
fitted models with these noisy signals was however found to be normal
distributed. These residuals most likely origins from a normal
distributed measurement noise. The noisy signals gave a large spread
between the lower and upper limits of the regressed parameter confidence
intervalls. But since the residuals were normal distributed, the mean
values of these parameter estimations gave a high accuracy when
evaluating with simulations. For the first model test, which was a very
long test in time, the cubic model gave a bit better accuracy. Also
using the regressed parameters to simulate a complete new dataset from
the second model test (a test not seen by the regression) gave good
accuracy for the two models.
For a shorter time span and amplitude span the linear model gives good
accuracy. For longer time and amplitude spans the cubic model seems to
be a better alternative. The parameter confidence intervalls of the
cubic model have larger spreadings, which will make direct comparison of
parameter values, for instance between two different ships, or two
different speeds more unreliable.
% Add a bibliography block to the postdoc
